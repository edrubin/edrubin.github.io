\documentclass[english,10pt,aspectratio=1610]{beamer}
% Choices for "aspectratio":
% - "1610": Sets aspect ratio to 16:10, and frame size to 160mm by 100mm.
% - "169": Sets aspect ratio to 16:9, and frame size to 160mm by 90mm.
% - "149": Sets aspect ratio to 14:9, and frame size to 140mm by 90mm.
% - "141": Sets aspect ratio to 1.41:1, and frame size to 148.5mm by 105mm.
% -  "54": Sets aspect ratio to 5:4, and frame size to 125mm by 100mm.
% -  "43": The default aspect ratio and frame size.
% -  "32": Sets aspect ratio to 3:2, and frame size to 135mm by 90mm.

\usepackage[sfdefault,light]{roboto}
\usepackage[T1]{fontenc}
\usepackage[activate={true,nocompatibility},final,tracking=true,kerning=true,spacing=true,factor=1100,stretch=10,shrink=10]{microtype}
% activate={true,nocompatibility} - activate protrusion and expansion
% final - enable microtype; use "draft" to disable
% tracking=true, kerning=true, spacing=true - activate these techniques
% factor=1100 - add 10% to the protrusion amount (default is 1000)
% stretch=10, shrink=10 - reduce stretchability/shrinkability (default is 20/20)
\microtypecontext{spacing=nonfrench}

\usepackage[latin9]{inputenc}
\usepackage{adjustbox, amsfonts, amsmath, amsthm, appendix, booktabs, chngcntr, dcolumn, dsfont, enumerate, float, geometry, graphicx, hyperref, mathtools, multicol, rotating, soul, subfiles, txfonts, type1cm, verbatim}

% Dummy text
\usepackage{lipsum}
\usepackage{xparse}
% Store a big set of sentences
\UnpackLipsum[1-100]
\ExplSyntaxOn
% Unpack \lipsumexp
\seq_new:N \g_lipsum_sentences_seq
\cs_generate_variant:Nn \seq_set_split:Nnn { NnV }
\seq_gset_split:NnV \g_lipsum_sentences_seq {.~} \lipsumexp
\NewDocumentCommand{\lipsumsentence}{>{\SplitArgument{1}{-}}O{1-7}}
 {
  \lipsumsentenceaux #1
 }
\NewDocumentCommand{\lipsumsentenceaux}{mm}
 {
  \IfNoValueTF { #2 }
   {
    \seq_item:Nn \g_lipsum_sentences_seq { #1 }.~
   }
   {
    \int_step_inline:nnnn { #1 } { 1 } { #2 }
     {
      \seq_item:Nn \g_lipsum_sentences_seq { ##1 }.~
     }
   }
 }
\ExplSyntaxOff

% Error in subcaption package
\usepackage{subcaption}
\captionsetup{compatibility=false}

% Line spacing
\usepackage{setspace}
\setstretch{1.3}

% LIGHTER GREYS: Define colors
\definecolor{my-very-light-grey}{HTML}{EEEEEE}
\definecolor{my-light-grey}{HTML}{C7C7C7}
\definecolor{my-mid-grey}{HTML}{9E9E9E}
\definecolor{my-dark-grey}{HTML}{7D7D7D}
\definecolor{my-red-pink}{HTML}{E40045}
\definecolor{my-purple}{HTML}{310769}
\definecolor{my-light-purple}{HTML}{9E17A0}
\definecolor{my-orange}{HTML}{FF6F00}

% % DARKER GREYS: Define colors
% \definecolor{my-very-light-grey}{HTML}{E0E0E0}
% \definecolor{my-light-grey}{HTML}{BDBDBD}
% \definecolor{my-mid-grey}{HTML}{9E9E9E}
% \definecolor{my-dark-grey}{HTML}{616161}
% \definecolor{my-red-pink}{HTML}{E40045}
% \definecolor{my-purple}{HTML}{310769}
% \definecolor{my-light-purple}{HTML}{9E17A0}
% \definecolor{my-orange}{HTML}{FF6F00}

% Links
\usepackage{hyperref}
\hypersetup{
  colorlinks=true,
  linkcolor=my-dark-grey,
  filecolor=my-dark-grey,
  urlcolor=my-dark-grey,
}
% Links have same style as other text
\urlstyle{same}

% Beamer theme
% \usetheme{default}
\usetheme{Pittsburgh}
\usecolortheme{beetle}
\usefonttheme{default}
% Remove navigation symbols
\setbeamertemplate{navigation symbols}{}

% Change theme title justification
\setbeamertemplate{frametitle}[default][left]

% LIGHTER GREYS: Change theme colors
\setbeamercolor{frametitle}{bg=white, fg=my-dark-grey}
\setbeamercolor{framesubtitle}{bg=white, fg=my-light-grey}
\setbeamercolor{normal text}{fg=black}
\setbeamercolor{background canvas}{bg=white}
\setbeamercolor{title}{fg=my-dark-grey}
\setbeamercolor{subtitle}{fg=my-light-grey}
\setbeamercolor{item projected}{bg=white, fg=my-dark-grey}
\setbeamercolor{footline}{fg=my-light-grey}
\setbeamertemplate{itemize item}{\color{my-light-grey}$\circ$}
%\setbeamertemplate{itemize subitem}{\color{my-light-grey}$\wasylozenge$}
\setbeamertemplate{itemize subitem}{\color{my-light-grey}$\bullet$}
\setbeamertemplate{itemize subsubitem}{\color{my-light-grey}$-$}
%\setbeamertemplate{footline}[frame number]

% % DARKER GREYS: Change theme colors
% \setbeamercolor{frametitle}{bg=white, fg=my-dark-grey}
% \setbeamercolor{framesubtitle}{bg=white, fg=my-light-grey}
% \setbeamercolor{normal text}{fg=black}
% \setbeamercolor{background canvas}{bg=white}
% \setbeamercolor{title}{fg=my-dark-grey}
% \setbeamercolor{subtitle}{fg=my-light-grey}
% \setbeamercolor{item projected}{bg=white, fg=my-mid-grey}
% \setbeamercolor{footline}{fg=my-light-grey}
% \setbeamertemplate{itemize item}{\color{my-very-light-grey}$\circ$}
% %\setbeamertemplate{itemize subitem}{\color{my-very-light-grey}$\wasylozenge$}
% \setbeamertemplate{itemize subitem}{\color{my-very-light-grey}$\bullet$}
% \setbeamertemplate{itemize subsubitem}{\color{my-very-light-grey}$-$}
% %\setbeamertemplate{footline}[frame number]

\usepackage{stmaryrd}
\newcommand*\myBullet{
  % \item[\color{my-mid-grey}\scalebox{1.1}{$\circ$}]}
  \item[\color{my-mid-grey}\scalebox{1.1}{$\shortrightarrow$}]}

% Tikz setup
\usepackage{tikz}
\usetikzlibrary{shapes.geometric, arrows, fit}
% Tikz styles go here...
% \tikzstyle{startstop}   = [rectangle, minimum width=3cm, minimum height=1cm, text centered, draw=white, fill=white]
% \tikzstyle{startstophi} = [my-red-pink, rectangle, minimum width=3cm, minimum height=1cm, text centered, draw=white, fill=white, font=\bf]
% \tikzstyle{blank}       = [rectangle, minimum width=0cm, minimum height=1cm, text centered, draw=white, fill=white]
% \tikzstyle{io}          = [rectangle, minimum width=0cm, minimum height=1cm, text centered, draw=white, fill=white]
% \tikzstyle{iohi}        = [my-red-pink, rectangle, minimum width=0cm, minimum height=1cm, text centered, draw=white, fill=white, font=\bf]
% \tikzstyle{arrow}       = [black!50, thick,->,>=stealth]
% \tikzstyle{arrowhi}     = [my-red-pink, very thick,->,>=stealth, font=\bf]
% \tikzstyle{dblarrow}    = [black!50, thick,<->,>=stealth]
% \tikzstyle{dblarrowhi}  = [my-red-pink, very thick,<->,>=stealth, font=\bf]


% For table formatting:
\newcommand{\ra}[1]{\renewcommand{\arraystretch}{#1}}
\newcolumntype{d}[1]{D{.}{.}{#1}}

% Italic i.e.
\newcommand{\ie}{\textit{i.e. }}
% Italic e.g.
\newcommand{\eg}{\textit{e.g. }}
% Italic et al.
\newcommand{\etal}{\textit{et al. }}
% Italic etc.
\newcommand{\etc}{\textit{etc. }}

% Define functions to change row colors in a table
\makeatletter
\def\zapcolorreset{\let\reset@color\relax\ignorespaces}
\def\colorrows#1{\noalign{\aftergroup\zapcolorreset#1}\ignorespaces}
\makeatother

% START HERE ------------------------------------------------------------

\title{Your clever, pun-ridden title}
\subtitle{Evidence from the subtitle that actually describes your paper}

\author[Names]{%
  \texorpdfstring{%
    \begin{columns}
      \column{.3\linewidth}
      \centering
      {Author One} \\ \textcolor{my-light-grey}{A School}
      \column{.3\linewidth}
      \centering
      {Author Two} \\ \textcolor{my-light-grey}{B School}
    \end{columns}
 }
 {Author One, Author Two}
}

\date{Today}

\begin{document}

\maketitle

%  Tikz setup for boxes
% Introduce a new counter for counting the nodes needed for circling
\newcounter{nodecount}
% Command for making a new node and naming it according to the nodecount counter
\newcommand\tabnode[1]{\addtocounter{nodecount}{1} \tikz \node  (\arabic{nodecount}) {#1};}
% Some options common to all the nodes and paths
\tikzstyle{every picture}+=[remember picture,baseline]
\tikzstyle{every node}+=[anchor=base,
minimum width=1.5cm,align=center,text depth=.25ex,outer sep=0.5pt]
\tikzstyle{every path}+=[thick, rounded corners]

\section{Your first section}

\begin{frame}{A slide title}{The slide's subtitle}
  \lipsumsentence[1-2] \\ \vspace{2em}
  \pause
  \lipsumsentence[3] \\ \vspace{2em}
  \pause
  \lipsumsentence[4]
  \[ \mathbf{y}_g = \mathbf{X}_g \boldsymbol{\beta} + \boldsymbol{\varepsilon}_g \ \]
\end{frame}

\begin{frame}{Several important points}{Three points}
\textbf{\lipsumsentence[5]}
  \begin{enumerate}
    \item \lipsumsentence[6]
    \item \lipsumsentence[7]
    \item \lipsumsentence[8]
  \end{enumerate}
\end{frame}

\section{Results}

\begin{frame}
\textcolor{my-dark-grey}{\large Results}
\end{frame}

\begin{frame}{Preliminary results}{Elasticities}
\begin{small}
  \vspace{-2em}
  \begin{table}[!htbp] \centering
  \begin{tabular}{@{\extracolsep{-6pt}}lcccccc}
   & \multicolumn{6}{c}{\textbf{Dependent variable:} Log$\left(q_{i,t}/\text{days}_{i,t}\right)$} \\
  \toprule
   & \multicolumn{3}{c}{{Marginal price}} & \multicolumn{3}{c}{{Average price}} \\
  \cmidrule(lr{1pt}){2-4} \cmidrule(lr{1pt}){5-7}
   & \tabnode{(\textbf{1})} & \tabnode{(\textbf{2})} & \tabnode{(\textbf{3})} & \tabnode{(\textbf{4})} & \tabnode{(\textbf{5})} & \tabnode{(\textbf{6})}\\
  \midrule \\[-1.1ex]
   Log$\left(p_{i,t}\right)$   & \tabnode{$-$0.2222$^{***}$} & \tabnode{$-$0.3333$^{***}$} & \tabnode{$-$0.4444$^{***}$} & \tabnode{$-$0.2323$^{***}$} & \tabnode{$-$0.2424$^{***}$} & \tabnode{$-$0.2525$^{***}$} \\ [-0.25ex]
   \quad \textit{instrumented} & \tabnode{(0.0222)} & \tabnode{(0.0333)} & \tabnode{(0.0444)} & \tabnode{(0.0555)} & \tabnode{(0.0444)} & \tabnode{(0.0333)} \\ [0.5ex]
   Covariate                   & & 0.1234$^{***}$  & 0.5678$^{***}$ & & 0.9999$^{***}$ & 0.0001 \\ [-0.25ex]
  \quad (thousands)            & & (0.0123) & (0.0567) & & (0.1111) & (0.1000) \\ [0.5ex]
  \midrule
  Simulated IV & F & F & T & F & F & T \\ [-0.25ex]
  FE$_1$       & T & T & T & T & T & T \\ [-0.25ex]
  FE$_2$       & T & T & T & F & F & F \\ [-0.25ex]
  \textit{N}   & \tabnode{1,234,567} & \tabnode{1,234,567} & \tabnode{1,234,567} & \tabnode{1,234,567} & \tabnode{1,234,567} & \tabnode{1,234,567} \\
  \bottomrule
  \end{tabular}
  \caption*{\raggedright\footnotesize Errors are two-way clustered within (1) $i$ and (2) $c(i)$ by $t$. Significance levels: *10\%, **5\%, ***1\%.}
  \end{table}
  % The overlayed Tikz boxes
  \begin{tikzpicture}[overlay]
    % Define the circle paths
    \only<2>\node[draw=my-red-pink,rounded corners = 0.5ex,fit=(1)(19),inner sep = 0pt] {};
    \only<3-7>\node[draw=my-red-pink,rounded corners = 0.5ex,fit=(7)(8)(9)(15),inner sep = 0pt] {};
    \only<4>\node[draw=my-purple,rounded corners = 0.5ex,fit=(7)(13),inner sep = 0pt] {};
    \only<5>\node[draw=my-purple,rounded corners = 0.5ex,fit=(8)(14),inner sep = 0pt] {};
    \only<6>\node[draw=my-purple,rounded corners = 0.5ex,fit=(9)(15),inner sep = 0pt] {};
    \only<7>\node[draw=my-purple,rounded corners = 0.5ex,fit=(10)(11)(12)(18),inner sep = 0pt] {};
  \end{tikzpicture}
\end{small}
\end{frame}

\begin{frame}
  \bigskip\bigskip
  \textcolor{black}{\large \textcolor{my-purple}{\textbf{Thank you!}}}
  \\ \bigskip\bigskip
  \textcolor{black}{Ed Rubin}
  \\
  \textcolor{my-mid-grey}{edward@berkeley.edu}
  \\
  \textcolor{my-mid-grey}{http://edrub.in}
\end{frame}

\end{document}
